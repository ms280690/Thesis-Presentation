%%%%%%%%%%%%%%%%%%%%%%%%%%%%%%%%%%%%%%%%%
% Beamer Presentation
% LaTeX Template
% Version 1.0 (10/11/12)
%
% This template has been downloaded from:
% http://www.LaTeXTemplates.com
%
% License:
% CC BY-NC-SA 3.0 (http://creativecommons.org/licenses/by-nc-sa/3.0/)
%
%%%%%%%%%%%%%%%%%%%%%%%%%%%%%%%%%%%%%%%%%

%----------------------------------------------------------------------------------------
%	PACKAGES AND THEMES
%----------------------------------------------------------------------------------------

\documentclass{beamer}

\mode<presentation> {

% The Beamer class comes with a number of default slide themes
% which change the colors and layouts of slides. Below this is a list
% of all the themes, uncomment each in turn to see what they look like.

%\usetheme{default}
%\usetheme{AnnArbor}
%\usetheme{Antibes}
%\usetheme{Bergen}
%\usetheme{Berkeley}
%\usetheme{Berlin}
%\usetheme{Boadilla}
%\usetheme{CambridgeUS}
%\usetheme{Copenhagen}
%\usetheme{Darmstadt}
%\usetheme{Dresden}
%\usetheme{Frankfurt}
%\usetheme{Goettingen}
%\usetheme{Hannover}
%\usetheme{Ilmenau}
%\usetheme{JuanLesPins}
%\usetheme{Luebeck}
%\usetheme{Madrid}
%\usetheme{Malmoe}
%\usetheme{Marburg}
%\usetheme{Montpellier}
\usetheme{PaloAlto}
%\usetheme{Pittsburgh}
%\usetheme{Rochester}
%\usetheme{Singapore}
%\usetheme{Szeged}
%\usetheme{Warsaw}

% As well as themes, the Beamer class has a number of color themes
% for any slide theme. Uncomment each of these in turn to see how it
% changes the colors of your current slide theme.

%\usecolortheme{albatross}
%\usecolortheme{beaver}
%\usecolortheme{beetle}
%\usecolortheme{crane}
%\usecolortheme{dolphin}
%\usecolortheme{dove}
%\usecolortheme{fly}
%\usecolortheme{lily}
\usecolortheme{orchid}
%\usecolortheme{rose}
%\usecolortheme{seagull}
%\usecolortheme{seahorse}
%\usecolortheme{whale}
%\usecolortheme{wolverine}

%\setbeamertemplate{footline} % To remove the footer line in all slides uncomment this line
%\setbeamertemplate{footline}[page number] % To replace the footer line in all slides with a simple slide count uncomment this line

%\setbeamertemplate{navigation symbols}{} % To remove the navigation symbols from the bottom of all slides uncomment this line
}

\usepackage{graphicx} % Allows including images
\usepackage{booktabs} % Allows the use of \toprule, \midrule and \bottomrule in tables

%---------------------------------------------------------------------------------------------
%%  This file is really -*-LaTeX-*-
%%             File:  thesis-macros.sty
%%          Created:  Sat Feb 20 23:33:41 2016
%%
%%      Description:  Commands particular to this thesis

\newsavebox\xXx
\newcommand{\xxx}[1]{\savebox\xXx{#1}#1\kern-1.0\wd\xXx\textcolor{red}{\rule[0.7ex]{\wd\xXx}{1pt}}%
	\kern-1.0\wd\xXx\textcolor{red}{\rule[0.25ex]{\wd\xXx}{1pt}}%
}
\newcommand{\replace}[2]{\xxx{#1}\,{\color{blue}#2}}
\let\yyy\replace

\newcommand{\progLang}[1]{\textsc{#1}}
\newcommand{\unknownLabel}[1]{\texttt{\large\color{dark-red}\slshape #1}}
\newcommand{\markWord}[1]{\unknownLabel{#1}\eref{text-kind}}
\newcommand{\chapterReference}[2]{\textcolor{blue}{\underline{\hyperref[#1]{#2}}}}

\definecolor{dark-purple}{rgb}{0.40,0.05,0.40}\relax
\definecolor{dark-red}{rgb}{0.50,0.00,0.00}\relax
\definecolor{dark-green}{rgb}{0.00,0.50,0.00}\relax
\definecolor{fuchsia}{rgb}{0.99,0.50,0.99}\relax
\newcommand*{\mehul}[1]{\textcolor{dark-red}{#1}}
\newcommand*{\david}[1]{\textcolor{dark-green}{#1}}
\newcommand\firstOfTwo[2]{#1}
\newcommand\secondOfTwo[2]{#2}
\newcommand{\enparen}[1]{\textup{\firstOfTwo()}#1\textup{\secondOfTwo()}}
\newcommand{\macroName}[1]{\texttt{\char`\\#1}}
\newcommand{\macroArg}[1]{\texttt{\char`\{#1\char`\}}}
\newcommand{\prologConstruct}[1]{\textcolor{brown}{\texttt{#1}}}
\newcommand{\haskellConstruct}[1]{\textcolor{magenta}{\texttt{#1}}}
\newcommand{\haskellClass}[1]{\haskellConstruct{#1}}
\newcommand{\haskellVar}[1]{\haskellConstruct{\textit{#1}}}
\newcommand{\languageConstruct}[1]{\textcolor{orange}{\texttt{#1}}}
\newcounter{butCounter}
\newcommand\butbut{\addtocounter{butCounter}{1}\edef\next{%
		\noexpand\endnote{This is sentence \# \arabic{butCounter} starting %
			with ``But''.}}\next}

\providecommand\codeLibrary[1]{\texttt{\bfseries #1}}
\providecommand\metaSyntacticVariable[1]{\textsl{\ttfamily #1}}
\providecommand\mSV{\metaSyntacticVariable}

%----------------------------------------------------------------------------------------
%	TITLE PAGE
%----------------------------------------------------------------------------------------
\title[]{Embedding  Programming Languages: \progLang{Prolog} in \progLang{Haskell}} % The short title appears at the bottom of every slide, the full title is only on the title page

\author{Mehul Solanki} % Your name
\institute[UNBC] % Your institution as it will appear on the bottom of every slide, may be shorthand to save space
{
	University of Northern British Columbia \newline \\ % Your institution for the title page
	\medskip
	\textit{solanki@unbc.ca} % Your email address
}
\date{\today} % Date, can be changed to a custom date

\begin{document}
	\setbeamertemplate{footline}[frame number]
	\setbeamertemplate{itemize items}[circle]
	\setbeameroption{show notes} % un-comment to see the notes
	\begin{frame}
		\titlepage % Print the title page as the first slide
		\note[item]{Good afternoon everybody. My name is Mehul Solanki. I am a graduate student in the computer science department 
		under  Dr. David Casperson. And today I would like to talk about embedding \progLang{Prolog} in \progLang{Haskell}.}
	\end{frame}

\begin{frame}[allowframebreaks=0.4]
\frametitle{Overview} % Table of contents slide, comment this block out to remove it
\tableofcontents % Throughout your presentation, if you choose to use \section{} and \subsection{} commands, these will automatically be printed on this slide as an overview of your presentation
\note[item]{We will being with a preliminaries, motivation and background along with an overview of the contributions.}
\note[item]{Next, we dive into the approaches to merge different programming languages and the languages themselves.}
\note[item]{We will also look at our methodology and the prototype implementations.}
\note[item]{And lastly, we sum up. So let's get cracking.}
\end{frame}

\clearpage

\section{Introduction}
\begin{frame}
\begin{itemize}
\item Programming and programming languages.
\note[item]{Beginning with programming languages, they are in the hundereds and thousands.}
\note[item]{They cater to different needs and requirements.}
\note[item]{They are classified depending on their properties.}


\item Declarative style of programming.
\note[item]{One such style is the declarative paradigm, where the describe the logic of a computation without the control flow.[Practical 
Advantages of Declarative Programming Lloyd, J.W]}
\note[item]{It further branches out into functional and logic programming.}
\end{itemize}
\end{frame}

\clearpage

\section{Motivation}
\begin{frame}
\begin{itemize}
\item Language classification.
\note[item]{Languages are classified into categories known paradigms depending on their characteristics.}
\note[item]{Languages from the same paradigm may exhibit different properties.} 

\item For example, \progLang{Haskell}(functional language) and \progLang{Prolog}(logic language).
\note[item]{\progLang{Haskell} is a functional programming language while \progLang{Prolog} is a logic programming language.}
\note[item]{Both of them are declarative in nature but work differently.}

\item Scope.
\note[item]{} 
\end{itemize}
\end{frame}

\clearpage

\section{Problem Statement}
\begin{frame}
\begin{itemize}
\item
\end{itemize}

\end{frame}

\clearpage

\section{Background}
\begin{frame}
\begin{itemize}
\item
\end{itemize}

\end{frame}


\clearpage

\section{Existing and Proposed Work}
\begin{frame}
\begin{itemize}
\item
\end{itemize}

\end{frame}


\clearpage

\section{Other Contributions}
\begin{frame}
\begin{itemize}
\item
\end{itemize}

\end{frame}


\clearpage

\section{Embedding a Programming Language into another Programming Language}
\begin{frame}
\begin{itemize}
\item
\end{itemize}

\end{frame}


\clearpage

\section{Multi Paradigm Languages (Functional Logic Languages)}
\begin{frame}
\begin{itemize}
\item
\end{itemize}

\end{frame}


\clearpage

\section{\progLang{Haskell}}
\begin{frame}
\begin{itemize}
\item
\end{itemize}

\end{frame}


\clearpage

\section{\progLang{Prolog}}
\begin{frame}
\begin{itemize}
\item
\end{itemize}

\end{frame}


\clearpage

\section{Related Concepts}
\begin{frame}
\begin{itemize}
\item
\end{itemize}

\end{frame}


\clearpage

\section{Functorizing a language}
\begin{frame}
\begin{itemize}
\item
\end{itemize}

\end{frame}


\clearpage

\section{Monadic unification}
\begin{frame}
\begin{itemize}
\item
\end{itemize}

\end{frame}


\clearpage

\section{Prototype 1}
\begin{frame}
\begin{itemize}
\item
\end{itemize}

\end{frame}


\clearpage

\section{Prototype 2}
\begin{frame}
\begin{itemize}
\item
\end{itemize}

\end{frame}


\clearpage

\section{Prototype 3}
\begin{frame}
\begin{itemize}
\item
\end{itemize}

\end{frame}


\clearpage

\section{Prototype 4}
\begin{frame}
\begin{itemize}
\item
\end{itemize}

\end{frame}


\clearpage

\section{Future Scope}
\begin{frame}
\begin{itemize}
\item
\end{itemize}

\end{frame}


\clearpage

\section{Conclusion}
\begin{frame}
\begin{itemize}
\item
\end{itemize}

\end{frame}


\clearpage

\section{Bibliography}
\begin{frame}[allowframebreaks]
	\frametitle{Bibliography}
	\setbeamertemplate{bibliography item}{[\theenumiv]}
	\nocite{*} 
	\bibliographystyle{plain} 
	\bibliography{Thesis-Presentation}
\end{frame}

\clearpage

%------------------------------------------------

\begin{frame}
\Huge{\centerline{The End}}
\end{frame}

%----------------------------------------------------------------------------------------

\end{document} 