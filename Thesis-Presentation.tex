%%%%%%%%%%%%%%%%%%%%%%%%%%%%%%%%%%%%%%%%%
% Beamer Presentation
% LaTeX Template
% Version 1.0 (10/11/12)
%
% This template has been downloaded from:
% http://www.LaTeXTemplates.com
%
% License:
% CC BY-NC-SA 3.0 (http://creativecommons.org/licenses/by-nc-sa/3.0/)
%
%%%%%%%%%%%%%%%%%%%%%%%%%%%%%%%%%%%%%%%%%

%----------------------------------------------------------------------------------------
%	PACKAGES AND THEMES
%----------------------------------------------------------------------------------------

\documentclass[hideothersubsections, t, aspectratio=1610]{beamer}

\mode<presentation> {

% The Beamer class comes with a number of default slide themes
% which change the colors and layouts of slides. Below this is a list
% of all the themes, uncomment each in turn to see what they look like.

%\usetheme{default}
%\usetheme{AnnArbor}
%\usetheme{Antibes}
%\usetheme{Bergen}
%\usetheme{Berkeley}
%\usetheme{Berlin}
%\usetheme{Boadilla}
%\usetheme{CambridgeUS}
%\usetheme{Copenhagen}
%\usetheme{Darmstadt}
%\usetheme{Dresden}
%\usetheme{Frankfurt}
%\usetheme{Goettingen}
%\usetheme{Hannover}
%\usetheme{Ilmenau}
%\usetheme{JuanLesPins}
%\usetheme{Luebeck}
%\usetheme{Madrid}
%\usetheme{Malmoe}
%\usetheme{Marburg}
%\usetheme{Montpellier}
\usetheme{PaloAlto}
%\usetheme{Pittsburgh}
%\usetheme{Rochester}
%\usetheme{Singapore}
%\usetheme{Szeged}
%\usetheme{Warsaw}

% As well as themes, the Beamer class has a number of color themes
% for any slide theme. Uncomment each of these in turn to see how it
% changes the colors of your current slide theme.

%\usecolortheme{albatross}
%\usecolortheme{beaver}
%\usecolortheme{beetle}
%\usecolortheme{crane}
%\usecolortheme{dolphin}
%\usecolortheme{dove}
%\usecolortheme{fly}
%\usecolortheme{lily}
\usecolortheme{orchid}
%\usecolortheme{rose}
%\usecolortheme{seagull}
%\usecolortheme{seahorse}
%\usecolortheme{whale}
%\usecolortheme{wolverine}

%\setbeamertemplate{footline} % To remove the footer line in all slides uncomment this line
%\setbeamertemplate{footline}[page number] % To replace the footer line in all slides with a simple slide count uncomment this line

%\setbeamertemplate{navigation symbols}{} % To remove the navigation symbols from the bottom of all slides uncomment this line
}

\usepackage{graphicx} % Allows including images
\usepackage{booktabs} % Allows the use of \toprule, \midrule and \bottomrule in tables
\usepackage{tikz}
\usepackage{comment}
\usepackage{minted}
\usepackage{pgfpages}
%---------------------------------------------------------------------------------------------
%%  This file is really -*-LaTeX-*-
%%             File:  thesis-macros.sty
%%          Created:  Sat Feb 20 23:33:41 2016
%%
%%      Description:  Commands particular to this thesis

\newsavebox\xXx
\newcommand{\xxx}[1]{\savebox\xXx{#1}#1\kern-1.0\wd\xXx\textcolor{red}{\rule[0.7ex]{\wd\xXx}{1pt}}%
	\kern-1.0\wd\xXx\textcolor{red}{\rule[0.25ex]{\wd\xXx}{1pt}}%
}
\newcommand{\replace}[2]{\xxx{#1}\,{\color{blue}#2}}
\let\yyy\replace

\newcommand{\progLang}[1]{\textsc{#1}}
\newcommand{\unknownLabel}[1]{\texttt{\large\color{dark-red}\slshape #1}}
\newcommand{\markWord}[1]{\unknownLabel{#1}\eref{text-kind}}
\newcommand{\chapterReference}[2]{\textcolor{blue}{\underline{\hyperref[#1]{#2}}}}

\definecolor{dark-purple}{rgb}{0.40,0.05,0.40}\relax
\definecolor{dark-red}{rgb}{0.50,0.00,0.00}\relax
\definecolor{dark-green}{rgb}{0.00,0.50,0.00}\relax
\definecolor{fuchsia}{rgb}{0.99,0.50,0.99}\relax
\newcommand*{\mehul}[1]{\textcolor{dark-red}{#1}}
\newcommand*{\david}[1]{\textcolor{dark-green}{#1}}
\newcommand\firstOfTwo[2]{#1}
\newcommand\secondOfTwo[2]{#2}
\newcommand{\enparen}[1]{\textup{\firstOfTwo()}#1\textup{\secondOfTwo()}}
\newcommand{\macroName}[1]{\texttt{\char`\\#1}}
\newcommand{\macroArg}[1]{\texttt{\char`\{#1\char`\}}}
\newcommand{\prologConstruct}[1]{\textcolor{brown}{\texttt{#1}}}
\newcommand{\haskellConstruct}[1]{\textcolor{magenta}{\texttt{#1}}}
\newcommand{\haskellClass}[1]{\haskellConstruct{#1}}
\newcommand{\haskellVar}[1]{\haskellConstruct{\textit{#1}}}
\newcommand{\languageConstruct}[1]{\textcolor{orange}{\texttt{#1}}}
\newcounter{butCounter}
\newcommand\butbut{\addtocounter{butCounter}{1}\edef\next{%
		\noexpand\endnote{This is sentence \# \arabic{butCounter} starting %
			with ``But''.}}\next}

\providecommand\codeLibrary[1]{\texttt{\bfseries #1}}
\providecommand\metaSyntacticVariable[1]{\textsl{\ttfamily #1}}
\providecommand\mSV{\metaSyntacticVariable}

%----------------------------------------------------------------------------------------
%	TITLE PAGE
%----------------------------------------------------------------------------------------
\title[]{Embedding  Programming Languages: \newline \progLang{Prolog} in \progLang{Haskell}} % The short title appears at the bottom of every slide, the full title is only on the title page

\author{Mehul Solanki} % Your name
\institute[UNBC] % Your institution as it will appear on the bottom of every slide, may be shorthand to save space
{
	University of Northern British Columbia \newline \\ % Your institution for the title page
	\medskip
	\textit{solanki@unbc.ca} % Your email address
}
\date{\today} % Date, can be changed to a custom date

\begin{document}
	\setbeamertemplate{footline}[frame number]
	\setbeamertemplate{itemize items}[circle]
	
	\setbeameroption{show notes} % un-comment to see the notes
	\setbeameroption{show notes on second screen=right}
	
	\begin{frame}
		\titlepage % Print the title page as the first slide
		\note[item]{Good afternoon everybody. My name is Mehul Solanki. I am a graduate student in the computer science department 
		under  Dr. David Casperson. And today I would like to talk about embedding \progLang{Prolog} in \progLang{Haskell}.}
	\end{frame}

%% ====================================================================-
%% ====================================================================-
\section{Introduction}


%% -------------------------------------------------------------------
\begin{frame}[allowframebreaks=1.3]
\frametitle{Introduction} % Table of contents slide, comment this block out to remove it
\fontsize{10}{7.2}\selectfont
\tableofcontents % Throughout your presentation, if you choose to use \section{} and \subsection{} commands, these will automatically be printed on this slide as an overview of your presentation
\note[item]{We will begin with languages and how they are combined.}
\note[item]{The problem this thesis contributes to.}
\note[item]{Some concepts and the prototype implementations.}
\note[item]{And lastly, we sum up with possible directions.} 
\note[item]{So let's get started.}
\end{frame}
%% -------------------------------------------------------------------




%% ====================================================================-
%% ====================================================================-
\section{Background}

%% -------------------------------------------------------------------
\begin{frame}{Overview}
To provide background, I will discuss:
\begin{itemize}
\note[item]{In this section we will talk about,}
\item Programming languages and \progLang{Haskell} and \progLang{Prolog}.
\note[item]{Programming languages and programming paradigms.}
\note[item]{along with the languages in question, \progLang{Haskell} and \progLang{Prolog},}

\item Classification.
\note[item]{How they are classified}

\item Programming paradigms and declarative programming.
\note[item]{Next paradigms and the declarative stlye of programming in particular,} 

\item Functional and logic programming.
\note[item]{and it's sub-paradigms functional and logic programming,}

\item Bringing programming languages together.
\note[item]{lastly the approaches for bringing programming languages together.}
\end{itemize}
\end{frame}
%% -------------------------------------------------------------------

%% ====================================================================-
\subsection{Programming Languages}
\begin{frame}{Programming Languages}
\begin{columns}[T]
    \begin{column}{.4\textwidth}
     \begin{block}{}
A programming language is an artificial language designed to communicate
instructions to a machine, particularly a computer. 
For example: \textsc{C}, \textsc{Java}.
    \end{block}
    \end{column}
    \begin{column}{.6\textwidth}
    \begin{block}{}
% Your image included here
\begin{figure}[H]
    \includegraphics[width=1\textwidth]{progLanguages.jpg}
    
    \caption{The Universe of Programming Languages}
 \end{figure}   
    \end{block}
    \end{column}
  \end{columns}
\note[item]{A programming language is used to communicate instructions to a computer.}
\note[item]{They are in the hundereds of thousands and the number keeps on increasing.}
\end{frame}

%% -------------------------------------------------------------------
\begin{frame}
\frametitle{\progLang{Haskell}}
 \note[item]{So \progLang{Haskell} is a functional language,} 
  \begin{columns}[T]
    \begin{column}{.7\textwidth}
     \begin{block}{}
\progLang{Haskell} is an advanced purely-functional programming language. In particular, it 
\begin{itemize}
\item is polymorphically statically typed;
\note[item]{which is statically typed meaning every expression in Haskell has a type which is determined at compile time.}

\item has type inference;
\note[item]{has type type inference meaning you don't have to explicitly write out every type in a Haskell program.}

\item is lazy;
\note[item]{is lazy meaning functions don't evaluate their arguments until required.}

\item and is purely functional.
\note[item]{and is purely funcitonal meaning every function in Haskell is a function in the mathematical sense (i.e., "pure").}
\end{itemize}

    \end{block}
    \end{column}
    \begin{column}{.3\textwidth}
    \begin{block}{}
% Your image included here
\begin{figure}
    \includegraphics[width=\textwidth]{haskelllogo.jpg} 
    \caption{\progLang{Haskell} Programming Language}
 \end{figure}   
    \end{block}
    \end{column}
  \end{columns}
\end{frame}
%% -------------------------------------------------------------------





%% -------------------------------------------------------------------
\begin{frame}
\frametitle{\progLang{Prolog}}
  \begin{columns}[T]
    \begin{column}{.6\textwidth}
     \begin{block}{}
\progLang{Prolog} is a general purpose logic programming language with over 20 distributions. It borrows its basic 
constructs from logic. However, it has an order is defined for both clauses in the program, and 
for goals in the body of the clause.
\note[item]{\progLang{Prolog} is a general purpose programming language in which programs consist of facts and rules used to solve a 
query.}
\note[item]{It has a number of distributions, \progLang{SWI Prolog} being one of the popular ones.}
    \end{block}
    \end{column}
    \begin{column}{.4\textwidth}
    \begin{block}{}
% Your image included here
\begin{figure}
    \includegraphics[width=\textwidth]{swipl.png} 
    \caption{\textsc{SWI Prolog} Distribution}
 \end{figure}   
    \end{block}
    \end{column}
  \end{columns}
\end{frame}
%% -------------------------------------------------------------------

%% ====================================================================-
\subsection{Programming Language Paradigms}
\begin{frame}
\frametitle{Programming Language Paradigms}
  \begin{columns}[T]
    \begin{column}{.4\textwidth}
     \begin{block}{}
A programming paradigm is a fundamental style of computer programming, a way of building the structure and elements of computer programs.
\\*For example, Object Oriented Programming is programming language paradigm.
    \end{block}
    \end{column}
    \begin{column}{.6\textwidth}
    \begin{block}{}
% Your image included here
\begin{figure}
    \includegraphics[width=1\textwidth]{programminglanguageparadigms.jpg} 
    \caption{Programming Paradigms}
 \end{figure}   
    \end{block}
    \end{column}
  \end{columns}
  \note[item]{The fundamental style or manner in which a computer program is built is known as a programming paradigm.}
  \note[item]{For example, the object oriented style of programming.}
\end{frame}

\begin{frame}
\frametitle{Classification}
  \begin{columns}[T]
    \begin{column}{.5\textwidth}
     \begin{block}{}
Programming languages are classified into paradigms depending on their characteristics and features. 
\\*For example, \progLang{Java} is an Object Oriented Programming Language.
    \end{block}
    \end{column}
    \begin{column}{.5\textwidth}
    \begin{block}{}
% Your image included here
\begin{figure}
    \includegraphics[width=\textwidth]{classification.png} 
    \caption{Classification of Programming Languages}
 \end{figure}   
    \end{block}
    \end{column}
  \end{columns}
  \note[item]{Depending on the characteristics exhibited by a language into paradigms.}
  \note[item]{\progLang{Java} is an object oriented programming language.}
\end{frame}



%% -------------------------------------------------------------------
\begin{frame}
\frametitle{Declarative Programming}
\begin{itemize}
\item Declarative style of programming, describe (declaratively) what to do and not (operationally) how to do it. 
\note[item]{http://www.eolss.net/sample-chapters/c15/e6-45-05-04.pdf}
\note[item]{One such style is the declarative paradigm, where we describe the logic of a computation without the control flow.[Practical 
Advantages of Declarative Programming Lloyd, J.W]}

\note[item]{It further branches out into functional and logic programming.}

\item Programming in a functional language consists of building definitions and using the computer to evaluate expressions.
\note[item]{A funcitonal language is all about building and applying functions.}

\item In logic programming, a program consists of a collection of statements expressed as formulas in symbolic logic. There are rules of 
inference from logic that allow a new formula to be derived from old ones, with the guarantee that if the old formulas are true, so is the 
new one.
\note[item]{Logic programming allows a new formula to be derived from the old ones. A logic programming consists of a set of facts and 
rules. And these are used to solve a query.}

\item For example, \progLang{Haskell} (functional language) and
  \progLang{Prolog} (logic language).
\note[item]{\progLang{Haskell} is a functional programming language while \progLang{Prolog} is a logic programming language.}
\note[item]{Both of them are declarative in nature but work differently.}

\end{itemize}
\end{frame}
%% -------------------------------------------------------------------

\begin{comment}
\begin{frame}
\frametitle{Functional Programming}
  \begin{columns}[T]
    \begin{column}{.6\textwidth}
     \begin{block}{}
Programming in a functional language consists of building definitions and using the computer to evaluate expressions. 
\\*For example, \progLang{Haskell}.
\note[item]{}
    \end{block}
    \end{column}
    \begin{column}{.4\textwidth}
    \begin{block}{}
% Your image included here
\begin{figure}
    \includegraphics[width=\textwidth]{function-machine.jpeg} 
    \caption{Function}
 \end{figure}   
    \end{block}
    \end{column}
  \end{columns}
\end{frame}
\end{comment}

%% -------------------------------------------------------------------
\begin{frame}[fragile]{Functional Programming}
\note[item]{A few more details on functional programming,}
\begin{itemize}
\item Functional programming is all about functions.
\note[item]{Fucntional programming works by applying mathematical functions to arguments to get results. A program is nothing but 
	functions which can also be passed to other functions as arguments.}

\item $\lambda$-calculi, is a formal system in mathematical logic for expressing computations.
\note[item]{$\lambda$-calculi, is a formal system in mathematical logic for expressing computations.}
\note[item]{The typed variant of $\lambda$-calculi puts restrictions on the type of data a function can work with.}

\item For example, \progLang{Haskell} uses the Damas-Hindley-Milner type system provides the ability to give a most general type to a function or program.
\note[item]{It is based on the Damas-Hindley-Milner type system which provides the most general type to a function or program.}
\begin{minted}[linenos]{haskell}
add :: Num a => a -> a -> a
add x y = x + y
\end{minted}

\note[item]{For example, the \haskellConstruct{add} function takes as input numbers integers and returns their sum as the result.}

\item Functional programs are mostly side-effect free.
\note[item]{Functional programs are side-effect free, meaning, the state is not modified in any way when the functions are executed.}
\note[item]{Our \haskellConstruct{add} function does not change the values of x or y and just returns their sum.}
\end{itemize}
\end{frame}
%% -------------------------------------------------------------------





%% ====================================================================-
\subsection{Combining Programming Languages}

%% -------------------------------------------------------------------
\begin{frame}
\frametitle{Embedding Languages}
\note[item]{This section deals with embedding a target language in to the host environement.}
\note[item]{The aim of this approach is to bring the target language functionalities to the host language.}

Implementing a language within another language.

\begin{itemize}
\item Foreign Function Interface (FFI)
 \\* A mechanism by which a program written in one programming language can make use of services written in another language. 
\note{A foreign function interface allows programs to cooperate with programs written with other languages.}
\\* For example, \progLang{Java} provides a mechanism to embed code from other languages using the \progLang{Java} Native Interface (JNI).
\note[item]{For exmple, \progLang{Java} provides the \progLang{Java} Native Interface to interact with programs written in other languages.}

\item Library or Module Extension
\\* Replicate the features and characteristics of the target language into the host. 
\\* For example, the \codeLibrary{SAX} library is an \progLang{XML} library extension for \progLang{Java}. 
\note[item]{The SAX parser can be used to process and use XML data.}
\end{itemize}

\end{frame}
%% -------------------------------------------------------------------

%% -------------------------------------------------------------------
\begin{frame}{Embedding \progLang{Prolog}}
\note[item]{This section deals with embedding a target language in the host environement.}
\note[item]{The aim of this approach to bring the target language functionalities to the host language.}
\begin{itemize}
\item Embedding \progLang{Prolog}.
\note[item]{\progLang{Prolog} is a very popular choice as a target language.}

\item Host languages : \progLang{Java, Lisp, Scheme}.
\note[item]{Implementations exist for languages like \progLang{Java, Lisp} and \progLang{Scheme}.}

\item \progLang{Prolog} in \progLang{Haskell} : 
\note[item]{A few \progLang{Prolog} implementations do exist for \progLang{Haskell}.}
\begin{itemize}
\item \codeLibrary{Mini Prolog} : A micro \progLang{Prolog}-like language for an older specification for \progLang{Haskell} 98.

\item \codeLibrary{prolog} : A \progLang{Prolog}-like interpreter which can be interacted through a REPL.
\end{itemize}
 \note[item]{None of these are currently under active development and provide limited funcitonality. }
 \note[item]{Though we use components from these implementations, they do not utilize the haskellian features that makes life a lot simpler as we shall see.}

\item Logic programming in \progLang{Haskell}.
\note[item]{For adding logic programming capabiliies to \progLang{Haskell} there exist,}
\begin{itemize}
\item Propositional logic,
\item Backtracking, and,
\item Unification.
\end{itemize}
\note[item]{libraries related to propositional logic, backtracking and unification.}
\end{itemize}
\end{frame}
%% -------------------------------------------------------------------




%% -------------------------------------------------------------------
\begin{frame}
\note[item]{In this section we talk about languages that exhibit properties from multiple paradigms.}
\frametitle{Merging Paradigms}
\begin{itemize}
\item Combining different programming paradigms or programming styles into a single environment resulting in a hybrid language.
\note[item]{Attemp to combine properties(many a times conflicting in nature) of languages and/or paradigms }

\item The idea of a multi paradigm language is to provide a framework in which programmers can work in a variety of styles, freely
intermixing constructs from different paradigms.
\note[item]{A multi-paradigm programming language is a programming language that supports more than one programming paradigm allowing the 
	programmer to work in a variety of styles.[https://developer.mozilla.org/ar/docs/multiparadigmlanguage.html]}

\item For example, \progLang{Scala} is an object functional programming language.
\note[item]{A notable example is \progLang{Scala}, object functional language based on \progLang{Java}. Simply put it is Java with strong 
	functional characteristics such as higher order functions. Twitter is written in \progLang{Scala}.}

\item Functional Logic programming languages.
\note[item]{are hybrid declarative languages exhibiting functional and logic programming languages.}

\item For example, \progLang{Curry}.
\note[item]{is a \progLang{Haskell} based functional logic programming language providing the choice operator for non deterministic
	operations along with two way pattern matching.}

\item Other examples, \progLang{Mercury}.
\note[item]{which goes the other way round, starting from \progLang{Prolog} and adding \progLang{Haskell} features.}

\end{itemize}

\end{frame}
%% -------------------------------------------------------------------


%% ====================================================================-
%% ====================================================================-
\section{The Problem}

%% ====================================================================-
\subsection{Language Selection}
%% -------------------------------------------------------------------
\begin{frame}
	\frametitle{Choosing a Language}
	\note[item]{The versatility of a programming language is how well it can adapt to a particular problem.}

 \begin{columns}[T]
  \begin{column}{0.5\textwidth}
   \begin{block}{General Purpose Language}
    \textcolor{dark-green}{Broad scope} but \textcolor{red}{problem needs to be moulded according to the capability of the language}.
    \note[item]{A truly general purpose programming language, GPL, is described which contains facilities for constructing (within the 
     language) new data types as well as facilities for operations performed upon them.[Jan V. Garwick. 1968. Programming Languages: GPL, a 
     truly general purpose language. Commun. ACM 11, 9 (September 1968), 634-638. DOI=http://dx.doi.org/10.1145/364063.364092]}
   \end{block}
  \end{column}

  \begin{column}{0.5\textwidth}
   \begin{block}{Special Purpose Language}
    \textcolor{red}{Limited scope} but \textcolor{dark-green}{easier to express the problem as the suitable capabilities are readily available}. 
 	 \note[item]{A domain specific language is a concise micro language that offers tools and functionalities focused on solving problems 
 	 in a particular domain.} 
   \end{block}
  \end{column}
 \end{columns}

\begin{itemize}
\item For example, "Clarissa", a voice user interface for browsing procedures on the International Space Station.
	\note[item]{For instance, \progLang{Prolog} is good for rule based systems. "Clarissa", a voice user interface for browsing space station procedures is written in SICStus
		Prolog[https://ti.arc.nasa.gov/m/pub-archive/999h/0999\%20\%28Rayner\%29.pdf].}
	\note[item]{Simply speaking, it is like "siri" for the International Space Station.}
\end{itemize}
 
\end{frame}
%% -------------------------------------------------------------------

\begin{comment}
	%% ====================================================================-
\subsection{Languages}
\begin{frame}{Languages}
\begin{itemize}
\item Selection
\note[item]{Choosing a language is quite a daunting task. There are costs related to migration, training among others.}

\item Replicating functionality.
\note[item]{One option is to replicate the target language funcitonality in the current environment, and that is what we will look at 
today. Replicating \progLang{Prolog}-like functionality in \progLang{Haskell}}
\note[item]{The result is something close to a \textit{haskellised} \progLang{Prolog}, a functional eDSl with logic programming 
capabilities.}

\end{itemize}

\end{frame}
\end{comment}

%% ====================================================================-
\begin{frame}
\frametitle{Programmer's Dilemma}
\note[item]{Choosing a language is quite a daunting task. There are costs related to migration, training among others.}
\note[item]{And not to mention the plethora of options.}
\begin{figure}
    \includegraphics[height = 0.5\textwidth, width=0.5\textwidth]{programming-languages_2.png} 
    \caption{The Graph of programming Languages}
 \end{figure} 
 \note[item]{The graph depicts the varoius languages and how they are influenced and/or related to each other.}
\end{frame}

%% ====================================================================-
\subsection{Problem Statement}
\begin{frame}
   \frametitle{Thesis Statement}
   \begin{block}{Thesis statement}
     The thesis aims to provide insights into merging two declarative
     languages namely, \progLang{Haskell} and \progLang{Prolog} by
     embedding the latter into the former and analyzing the result of
     doing so\dots\,.
     \note[item]{\textcolor{red}{Read out Thesis statement.}}
   \end{block}
        \begin{itemize}
      \item Replicating functionality.
      \note[item]{One option is to replicate the target language funcitonality in the current environment,}
      
      \item \progLang{Prolog}-like functionality in \progLang{Haskell}.
      \note[item]{, and that is what we will look at today. Replicating \progLang{Prolog}-like functionality in \progLang{Haskell}}
      
      \item \textit{haskellised} \progLang{Prolog}-like eDSL
      \note[item]{The result is something close to a \textit{haskellised} \progLang{Prolog}, a functional eDSL with logic programming 
         capabilities.}
   \end{itemize}
\end{frame}

%% ====================================================================-
%% ====================================================================-
\section{Additional Concepts}
%% -------------------------------------------------------------------
\subsection{Monads}
\begin{frame}
	\frametitle{Monads}
\begin{itemize}
\item Monads in \progLang{Haskell} can be thought of as composable computation descriptions. They,
\note[item]{Monads are composable computation builders.}
\begin{itemize}
\item separate composition and computation execution time line,
\note[item]{separate how computations are composed and how they will be executed.}

\item carry and pass around data(state).
\note[item]{they hold results of the computation.}
\end{itemize}
\note[item]{This allows us to manipulat the control flow.}

\note[item]{This lends monads to supplementing pure calculations with features like I/O, common environment or state, etc.}

\item A monad is a structure that represents computations defined as sequences of steps.
\note[item]{Computations are a sequence of steps chained together.}

\item The monadic bubble.
\note[item]{Computations that produce side effects are carried out in a monad, a bubble, which prevents the outer state to get affected 
and hence the computation remains.}

\end{itemize}

\end{frame}
%% -------------------------------------------------------------------
	
%% -------------------------------------------------------------------
\subsection{Pattern Matching}
\begin{frame}[fragile]

\frametitle{Pattern Matching}
\begin{itemize}
\item In pattern matching, we attempt to match values against patterns and, if so desired, bind variables to successful matches.
\note[item]{In pattern matching, we matry and match values against patterns to bind values to variables.}
\note[item]{Pattern matching can either fail, succeed or diverge.}

\item Consider the example of the Fibonacci series in \progLang{Haskell}.
\begin{minted}[linenos]{haskell}
fib 0 = 0
fib 1 = 1
fib n = fib (n-1) + fib (n-2)
\end{minted}
\note[item]{\progLang{Haskell} selects the appropriate computation depending on the result of matching the value of n to the patterns one by one.}

\end{itemize}

\end{frame}
%% -------------------------------------------------------------------

%% -------------------------------------------------------------------
\subsection{Unification}
\begin{frame}[fragile]
\frametitle{Unification}
\begin{itemize}
\item The way in which \progLang{Prolog} matches two terms is called unification. 
\note[item]{The idea is similar to that of unification in logic: we have two terms and we want to see if they can be made to represent the same structure.}
\begin{itemize}
\item If term1 and term2 are constants, then term1 and term2 unify if and only if they are the same atom, or the same number.
\note[item]{Constants are unified if they are the same.}
\item If term1 is a variable and term2 is any type of term, then term1 and term2 unify, and term1 is instantiated to term2 . Similarly, if term2 is a variable and term1 is any type of term, then term1 and term2 unify, and term2 is instantiated to term1 . (So if they are both variables, they’re both instantiated to each other, and we say that they share values.)
\note[item]{If one of the terms is a variable then it is instantiated to the other term.}

\item If term1 and term2 are complex terms, then they unify if and only if:
They have the same functor and arity, and
all their corresponding arguments unify, and
the variable instantiations are compatible.
\note[item]{For nested terms, we match the heads and the number of arguments before finding a unifier.}

\item Two terms unify if and only if it follows from the previous three clauses that they unify.
\end{itemize}
\end{itemize}
\end{frame}

%% -------------------------------------------------------------------
\begin{frame}[fragile] 
\frametitle{Unification Examples} 
\note[item]{Let us take a look at a few examples,}
\begin{itemize}
\item Unifying variables,

\begin{minted}[linenos]{prolog} 
(X,2) = (1,Y). 
X = 1. 
Y = 2.  
\end{minted}
\note[item]{In the first example, \prologConstruct{X} is instantiated to 1 and \prologConstruct{Y} to 2.}

\item  Unifying complex terms,

\begin{minted}[linenos]{prolog} 
k(s(g),Y) = k(X,t(k)). 
X = s(g) 
Y = t(k) 
\end{minted} 
\note[item]{The first check is to match the head of the two terms \prologConstruct{k}, next the arity and finally \prologConstruct{X} and
\prologConstruct{Y} are instantiated.}

\item  Unification is not always easy\dots

\begin{minted}[linenos,gobble=2]{prolog}
  h(f(W),S,g(U,y)) = h(U,V,g(f(x),V))
  W=x
  U=f(x)
  S=V=y
\end{minted}

\end{itemize}
	
\end{frame} 

%% ====================================================================-

\begin{comment}
\subsection{Motivation}
\begin{frame}{Motivation}
\frametitle{(to be discarded ......)}
\begin{itemize}
\item Language classification.
\note[item]{Languages are classified into categories known paradigms depending on their characteristics.}
\note[item]{Languages from the same paradigm may exhibit different properties.} 

\item For example, \progLang{Haskell}(functional language) and \progLang{Prolog}(logic language).
\note[item]{\progLang{Haskell} is a functional programming language while \progLang{Prolog} is a logic programming language.}
\note[item]{Both of them are declarative in nature but work differently.}

\item Scope.
\note[item]{The versatility of a programming language is how well it can adapt to a particular problem.}
\note[item]{For instance, \progLang{Prolog} is good for rule based systems. "Clarissa", a voice user interface for browsing space station procedures is written in SICStus
	Prolog[https://ti.arc.nasa.gov/m/pub-archive/999h/0999\%20\%28Rayner\%29.pdf].}
\end{itemize}
\end{frame}
\end{comment}





%% ====================================================================-
\begin{comment}
\subsection{Discarded Scope}
\begin{frame}{Scope}
\frametitle{(to be discarded ......)}
\begin{itemize}
\item General purpose programming languages.
\note[item]{A truly general purpose programming language, GPL, is described which contains facilities for constructing (within the 
language) new data types as well as facilities for operations performed upon them.[Jan V. Garwick. 1968. Programming Languages: GPL, a 
truly general purpose language. Commun. ACM 11, 9 (September 1968), 634-638. DOI=http://dx.doi.org/10.1145/364063.364092]} 

\item Domain specific programming languages.
\note[item]{A domain specific language is a concise micro language that offers tools and functionalities focused on solving problems ona  
particular domain.} 
\end{itemize}
\end{frame}
\end{comment}

%% ====================================================================-
%% ====================================================================-
\section{What was done}

\begin{frame}
\frametitle{Previously Existing Work}
\note[item]{Here we will look at the existing work,}
\begin{itemize}
\item Implementations.
\note[item]{Few exist. Incomplete and lack pratical features.}
\begin{itemize}
\item \codeLibrary{Mini Prolog} : \progLang{Prolog}-interpreter with support for variable search strategies.
\note[item]{micro \progLang{Prolog}-like language consisting only of atoms and variables. The interpreter can work with multiple strategies predetermined at compile time.}

\item \codeLibrary{prolog} : A \progLang{Prolog} interpreter written in \progLang{Haskell}.
\note[item]{is a more complete implementation consusting of \prologConstruct{cuts} and \prologConstruct{fails}.}

\item \progLang{Curry} : A functional logic programming based on \progLang{Haskell}
\note[item]{Works on the principles of residuation and narrowing to unify terms.}
\end{itemize}

\item Literature.
\note[item]{The literature mainly focuses on,}
\begin{itemize}
\item Translating \progLang{Prolog} predicates into a \progLang{Haskell} function.
\note[item]{A series of papers provide useful insights into replicating \progLang{Prolog}-like clauses in \progLang{Haskell} though none
	have accompanying implementations.}

\item Passing state of a computation.
\note[item]{state of each step in computation is passed on as a stream of answer substitutions,}

\item Implementing a typed functional logic programming language.
\note[item]{To get full static typing we need to use the \progLang{Haskell} extensions of quantified types and the \haskellConstruct{ST 
Monad}.}
\end{itemize}

\item Libraries.
\begin{itemize}
\item \codeLibrary{unification-fd} : Generic unification algorithms. 
\note[item]{generic unification algorithms which are imperative in nature.}

\item \codeLibrary{logict} : A continuation-based, backtracking, logic programming monad.
\note[item]{which allows us to add backtracking computations to any Haskell monad}

\end{itemize}

\end{itemize}

\end{frame}



%% ====================================================================-
%% ====================================================================-
\section{What we did}

%% -------------------------------------------------------------------
\begin{frame}{What we did Overview}
	
	\frametitle{What we did Overview}
\begin{itemize}
\item Literature review
  \begin{itemize}
  \item We reviewed literature on embedded languages (Chapter~5)
  \item We reviewed literature on merging programming
    languages (Chapter~6)
  \end{itemize}
\item Improvements,
\begin{itemize}
        \item Practical features.
	\note[item]{Firstly, we added practical features a \progLang{Prolog} distribution might, these being \prologConstruct{cut} and
		\prologConstruct{fail}.}
	
	\item \progLang{Prolog} in \progLang{Haskell}.
	\note[item]{Meaning, one can write a program in our eDSl just like one would write a \progLang{Haskell} program.}
\end{itemize}

\item Other Contributions,
\note[item]{In this section we will talk about, the contributions which are not neccessary an improvement on the current work but a new
	direction for solving the problem of embedding languages.}
\begin{itemize}
	\item Modular \textit{functorizing} mechanism.
	\note[item]{We open up the language to accomodate meta syntactic variables, custom quantifiers and logic.}
	
	\item Modular \textit{monadic} unification procedure.
	\note[item]{this is to leverage imperative unification algorithms.}
	
	\item Basic working \progLang{Prolog} implementation.
	\note[item]{}
	
	\item Variable search strategies.
	\note[item]{Independent of how the solution search happens our implementation must work.}
	
	\item Embedding IO in an eDSL. 
	\note[item]{Modularizing the interpreter in a manner so that each stage must only deal with a particular aspect of the program.}
	
\end{itemize}
\
\end{itemize}

\end{frame}
%% -------------------------------------------------------------------


%% ====================================================================-
\begin{comment}
	\subsection{Improvements}
\begin{frame}{Improvements}
\note[item]{The improvements,}
\begin{itemize}
\item Practical features.
\note[item]{Firstly, we added practical features a \progLang{Prolog} distribution might, these being \prologConstruct{cut} and
\prologConstruct{fail}.}

\item \progLang{Prolog} in \progLang{Haskell}.
\note[item]{Meaning, one can write a program in our eDSl just like one would write a \progLang{Haskell} program.}
\end{itemize}
\end{frame}

\end{comment}

%% ====================================================================-
\begin{comment}
	\subsection{Other Contributions}
\begin{frame}{Other Contributions}
\note[item]{In this section we will talk about, the contributions which are not neccessary an improvement on the current work but a new
direction for solving the problem of embedding languages.}
\begin{itemize}
\item Modular \textit{functorizing} mechanism.
\note[item]{We open up the language to accomodate meta syntactic variables, custom quantifiers and logic.}

\item Modular \textit{monadic} unification procedure.
\note[item]{this is to leverage imperative unification algorithms.}

\item Basic working \progLang{Prolog} implementation.
\note[item]{}

\item Variable search strategies.
\note[item]{Independent of how the solution search happens our implementation must work.}

\item Embedding IO in an eDSL. 
\note[item]{Modularizing the interpreter in a manner so that each stage must only deal with a particular aspect of the program.}

\end{itemize}
\end{frame}

\end{comment}

%% ====================================================================-
\subsection{Prototype 1}

\begin{frame}[fragile=singleslide]
	\frametitle{{Functorizing a language}}
	\begin{itemize}
		\item Consider the following recursive grammar,
		
		
		$1.  S \rightarrow aSb$
		
		
		$2. S \rightarrow ba$
		
		Example term : $ S \Rightarrow aSb \Rightarrow aaSbb \Rightarrow aababb $
		
		\note[item]{A recursive grammar locks you into the language and does not allow you introduce external ??quantifiers and logic??}
		
		\item Closed language,
		
		\begin{minted}[linenos]{haskell}
		data Term = Struct Atom [Term]
		| Var VariableName
		| Wildcard
		| Cut Int deriving (Eq, Data, Typeable)
		\end{minted}
		\note[item]{Our expressions can only be  built using \progLang{Term} and nothing else.}
		
		\item Open language,
		\begin{minted}[linenos]{haskell}
		data FlatTerm a = Struct Atom [a]
		|  Var VariableName
		|  Wildcard
		|  Cut Int deriving (Show, Eq, Ord)
		\end{minted}
		\note[item]{On the other hand, due to the addition of the type variable \haskellConstruct{a}, functionality can be injected into the language. }
		
	\end{itemize}
	
\end{frame}

\begin{frame}[fragile=singleslide]
	\frametitle{{Functorizing a language}}
	\begin{itemize}
		
		\item Manual extension,
		\begin{minted}[linenos]{haskell}
		data Term = Struct Atom [Term]
		| Var VariableName
		| Wildcard
		| Cut Int
		| New_Constructor_1 .........
		| New_Constructor_2 ......... 
		deriving (Eq, Data, Typeable, .....)
		\end{minted}
		\note[item]{A manual extension would result in modying the base language and the operations related to it.}
		
		\item Functorized Extension,
		\begin{minted}[linenos]{haskell}
		data Extended f = New_Constructor_1 ...
		| New_Constructor_2 ....
		| Base (f (Extended f)) 
		deriving (Eq, Data, Typeable, .....)
		\end{minted}
		\note[item]{Meanwhile, for the open language, a quantifier such as \haskellConstruct{Extended} can be used. }
	\end{itemize}
\end{frame}

\begin{frame}
	\frametitle{Monadic Unification}
	\note[item]{Here, we make use of the \codeLibrary{unification-fd} library }
	\begin{itemize}
		\item \codeLibrary{unification-fd} compatible
		\note[item]{Here, we make use of the \codeLibrary{unification-fd} library for providing the generic unification algorithm.}
		
		\item \textit{Unifiable} language.
		\note[item]{We create the necessary instances and make our language unifiable.}
		
		\item Convert the language expressions,    
		\note[item]{Convert the terms into \haskellConstruct{UTerm}}
		
		\item Extract and convert variables to become \haskellConstruct{State} compatible,
		\note[item]{the language varaibles are comnverted to state varaibles which store, pass and mutate values.}
		
		\item Carry out unification in the \haskellConstruct{Binding Monad}.
		\note[item]{The variables are bound to values representing substituions}
		
		\item Re translate substitutions.
		\note[item]{Extract state from the monad and translate the substitutions to the original language.}
	\end{itemize}
	
\end{frame}

\begin{frame}{Prototype 1}
\begin{figure}[H]
  \includegraphics[width=1\textwidth, height=0.8\textheight]{Prototype-1-architecture.jpeg}
  \caption{Architecture of Prototype 1}
  \label{fig:proto1-arch}
\end{figure}
\note[item]{Let us take a look at the architecture for Prototype 1.}
\note[item]{We adopt a sample \progLang{Prolog}-like language}
\note[item]{The language is open up by introdcing a type variable.}
\note[item]{Make the language unifiable.}
\note[item]{Perform the unification in the monad.}
\note[item]{State is used to extract the substituion and return the result.}
\end{frame}


%% ====================================================================-
\subsection{Prototype 2}

\begin{frame}
\frametitle{Contributions}
\note[item]{For a logic programming language,}
\begin{itemize}
\item A query resolver matches a query to the rules in the knowledge base and generate a list of goals which when achieved a result is returned.
\note[item]{A query resolver matches a query to the rules in the knowledge base and generate a list of goals which when achieved a result is returned.}

\item A query resolver consists of,
\begin{itemize}
\item a scheduling policy,
\note[item]{ defines how the additions and deletions of goals from the resolvent is performed by the interpreter.}

\item a search strategy,
\note[item]{selecting possible alternatives while searching through a solution space.}

\item unification.
\note[item]{unify the terms in question to obtain a result.}
\end{itemize}  

\item Prototype 1 is only unification.
\note[item]{Prototype 1 is only unification.}

\item Prototype 2 is a \progLang{Prolog}-like interpreter.
\note[item]{The base implementation is taken from an already existing library for \progLang{Prolog} in \progLang{Haskell} and modified to 
accommodate an open grammar for the language and monadic unification.}
\note[item]{This prototype serves as a test to prove the modularity and flexibility of the procedures from prototype 1.}

\end{itemize}
\end{frame}

\begin{frame}{Prototype 2}
\begin{figure}[H]
  \includegraphics[width=1\textwidth, height=0.75\textheight]{Prototype-2-architecture.jpeg}
  \caption{Architecture of Prototype 2}
  \label{fig:proto1-arch}
\end{figure}
\note[item]{The procedure remain quite similar.}
\note[item]{Open up the language.}
\note[item]{Make it unifiable.}
\note[item]{Carry out the unification procedure.}
\note[item]{Extract and reconvert the results.}
\note[item]{Since this process takes place within the interpreter, the results are pushed back and the process continues.}
\end{frame}


%% ====================================================================-
\subsection{Prototype 3}

\begin{frame}
\frametitle{Contributions}
\note[item]{Prototype 2 demonstrated how the procedure can be applied to recursive grammars and carry out unification irrespective of the scheduling policy of the interpreter.}
\begin{itemize}
\item Variable search strategy,
\note[item]{In Prototype 3 we work with variable search strategies.}
\note[item]{The base implementation is based on a \progLang{Prolog}-like interpreter for \progLang{Haskell 98}.}

\note[item]{We work with 3 different search strategies,}
\begin{itemize}
\item Stack Engine,

\item Pure Engine, and,

\item Andorra Engine.
\end{itemize}

\end{itemize}
\end{frame}

\begin{frame}{Prototype 3}
\begin{figure}[H]
  \includegraphics[width=0.75\textwidth, height=0.75\textheight]{Prototype-3-architecture.jpeg}
  \caption{Architecture of Prototype 3}
  \label{fig:proto1-arch}
\end{figure}
\note[item]{The procedure for Prototype 3 remains the same.}
\note[item]{The query resolver takes an additional input which is the search strategy at compile time.}
\end{frame}


%% ====================================================================-
\subsection{Prototype 4}

\begin{frame}
\frametitle{Contributions}

\begin{itemize}
\item Embedding input-output capabilities to a domain specific language.
\note[item]{Here we focus on embedding IO.}

\item Constructors for IO operations in the abstract syntax.
\note[item]{By defining constructors for IO operations in the language itself.}
\note[item]{one reason why this can be done is that funcitons in \progLang{Haskell} are first class citizens; they are data.}

\item Interpreted program is pure.
\note[item]{upon interpretation the program is still pure as no actions has been executed.}

\item Two-stage interpretation strategy.
\note[item]{Because of this we can have a two stage interpretation strategy.}
\end{itemize}
\end{frame}


\begin{frame}{Prototype 4}
\begin{figure}[H]
  \includegraphics[width=1\textwidth]{Prototype-4-architecture.jpeg}
  \caption{Architecture of Prototype 4}
  \label{fig:proto1-arch}
\end{figure}
\note[item]{Consider Protoype 4.}
\note[item]{A program is written in a DSL is interpreted into a Result data type }
\note[item]{This consists of IO operations as constructors.}
\note[item]{The result is then run in the IO Monad(stage 2) producing the desired effects.}
\note[item]{So if the program does not consist of any IO operations the procedure remains pure through out.}
\note[item]{If it does, no problem the execution takes place in a monad anyways.}
\end{frame}


%% ====================================================================-
%% ====================================================================-
\section{What remains to be done}

\begin{frame}{Future Scope}
\begin{itemize}
\item Quasi quoter with anti-quotation.
\note[item]{Quasiquoting allows programmers to use custom, domain-specific syntax to construct fragments of their program.} 
\note[item]{Antiquotation allows interchangeable usage of programming constructs of different languages in the same expression.}

\item Variable run time search strategy.
\note[item]{Currently prorotype 3 works with multiple search strategies but they need to specified at compile time. An addition would be enable the selection at run time.}

\item Additional database capabilities similar to \progLang{SWI Prolog}.
\note[item]{SWI \progLang{Prolog}, a popular \progLang{Prolog} distribution different database mechanisms.}
\begin{itemize}
\item \prologConstruct{assert/retract} database manipulation.
\note[item]{}

\item recorded database.
\note[item]{database terms are associated with a key}
\end{itemize}

\item Multi type variable language.
\note[item]{Currently the prototypes work with grammars with a single type. A multi type variable language would allow its contructotrs 
to be of different types.}

\item Prototype 4 additions and extension.
\note[item]{Additions and extension could be made to in the form of a hybrid constructor representing pure and impure operations.}
\end{itemize}

\end{frame}

%% ====================================================================-
%% ====================================================================-
\section{Conclusion}
%% ---------------------------------------------------------------------
\begin{frame}{Conclusion}
\note[item]{In conclusion, we set out to explore the various approaches for bringing languages together}
\begin{itemize}
\item Embedded domain specific language in \progLang{Haskell}.
\note[item]{support for eDSL's \progLang{Haskell}}

\item \progLang{Prolog}-like language which is closer to a \progLang{Prolog} distribution.
\note[item]{A more "complete" \progLang{Prolog} in \progLang{Haskell}.}

\item Prototype implementations.
\note[item]{We tested out methodologies for replicating functionalities with some prototypes.}

\item \progLang{Haskell}, an effective tool for embedding domain specific languages.
\note[item]{\progLang{Haskell}, an effective tool for embedding domain specific languages.}
\end{itemize}

\end{frame}
%% ---------------------------------------------------------------------

%% ---------------------------------------------------------------------
\begin{frame}{The End}
\begin{center}
\Huge Thank you!
\end{center}
\note[item]{That concludes the presentation.}
\note[item]{Ladies and gentlemen, thank you for your patience.}
\end{frame}
%% ---------------------------------------------------------------------

%% ---------------------------------------------------------------------
\begin{frame}{The End}
\begin{center}
\Huge Questions?
\end{center}
\note[item]{And questions please.}
\end{frame}
%% ---------------------------------------------------------------------

%% ---------------------------------------------------------------------
\begin{frame}
\Huge{\centerline{The End}}
\end{frame}
%% ---------------------------------------------------------------------

\begin{comment}
\section{Bibliography}
\begin{frame}[allowframebreaks]
	\frametitle{Bibliography}
	\setbeamertemplate{bibliography item}{[\theenumiv]}
	\nocite{*} 
	\bibliographystyle{plain} 
	\bibliography{Thesis-Presentation}
\end{frame}
\end{comment}

\clearpage



%----------------------------------------------------------------------------------------

\end{document} 
